% TEX compiler = lualatex (Verifique no Menu se está habilitado) %
%
%----------------Escolha uma dos estilos abaixo------------------%
%%%%%%%%%%%%%%%%%%%%%%%%%%%%%%%%%%%%%%%%%%%%%%%%%%%%%%%%%%%%%%%%%%
\documentclass[Portugues,Final, oneside]{comandos} % relatorio
%\documentclass[Portugues,Final,TwoSide]{comandos} % livro
%%%%%%%%%%%%%%%%%%%%%%%%%%%%%%%%%%%%%%%%%%%%%%%%%%%%%%%%%%%%%%%%%%
%                 Dúvidas acima? Leia abaixo:
% Opções: Portugues/Ingles/Espanhol, Final/Draft, TwoSide/oneside
% - Primeiramente, escolha: Portugues ou Ingles ou Espanhol.
% - Para a versão final do texto, acrescente a palavra "Final":
%\documentclass[Ingles,Final]{configuracoes}

% - Há dois modos de impressão configurados, Oneside - para impresso
% estilo relatório (ou pdf para visualização na web ou computador),
%e o estilo TwoSide - apropriado para impressão como um livro.

% - Para uma compilação mais rápida, utilize a opção "Draft":
%\documentclass[Portugues,Draft]{configuracoes}

% - Adicione também a opção "noFig" para gerar arquivos mais leves:
%\documentclass[Portugues,Draft,noFig]{configuracoes}

% - Para gerar uma avaliacao antiplagio pelo Turnitin:
%\documentclass[Portugues,Final,Turnitin]{configuracoes}

%%%%%%%%%%%%%%%%%%%%%%%%%%%%%%%%%%%%%%%%%%%%%%%%%%%%%%%%%%%%%%%%%%
% Configurações gerais do documento: (não edite)
\addbibresource{bibliografia.bib}
\setlength{\headheight}{14.5pt}

\usetikzlibrary{calc}

% Estilização dos Títulos de Capítulo e Apêndice:
\titleformat{\chapter}[display]
  {\normalfont\LARGE\bfseries}
  {\fontSourceSans\bfseries\color{dtecs_titulo_azul}\LARGE\chaptertitlename~\thechapter}
  {-10pt}
  {\huge}

% Definições para ficha catalográfica
\makeatletter
\StrBefore{\@autor}{ }[\@autorprenome]
\StrBehind{\@autor}{ }[\@autorsobrenome]
\makeatother

% Definições do glossário
\loadglsentries{Elementos/glossario}

%%%%%%%%%%%%%%%%%%%%%%%%%%%%%%%%%%%%%%%%%%%%%%%%%%%%%%%%%%%%%%%%%%
%            Preencha os dados na sequência abaixo:
%%%%%%%%%%%%%%%%%%%%%%%%%%%%%%%%%%%%%%%%%%%%%%%%%%%%%%%%%%%%%%%%%%
\begin{document} 
% Escolha entre autor ou autora:
\autor{Nome do autor}
%\autora{Nome da Autora}

% Comandos para a ficha catalográfica (formato ABNT)
\autorprenome{Prenome do autor}
\autorsobrenome{Ultimo Nome}
% O mesmo para o orientador:
\orientadorprenome{Nome Completo do}
\orientadorsobrenome{Orientador}

% Sempre deve haver um título em português:
\titulo{Título da Dissertação ou Tese em Português}
\subtitulo{Subtítulo da Dissertação ou Tese} % Se houver

% Se a língua for o inglês ou o espanhol defina:
%\title{The Dissertation or Thesis Title in English or Spanish}

% --- PALAVRAS-CHAVE ---
\palavraschave{Palavra1}{Palavra2}{Palavra3} %----ESCREVA AQUI----%

% Escolha entre orientador ou orientadora e inclua os títulos:
\orientador{Prof. Dr. Nome do Orientador}
%\orientadora{Profa. Dra. Nome da Orientadora}

% Escolha entre coorientador ou coorientadora, se houver:
%\coorientador{Prof. Dr. Eng. Lic. Nome do Co-Orientador}
%\coorientadora{Prof. Dra. Eng. Lic. Nome da Co-Orientadora}

% Escolha entre uma das seis opções a seguir (comente as demais):
%\tcc                   % para Trabalho de Conclusão de Curso.
%\qualificacaoMestrado  % Para textos de qualificação de mestrado.
%\qualificacaoDoutorado % Para textos de qualificação de doutorado.
\mestrado               % para Dissertação de Mestrado.
%\doutorado             % para Tese de Doutorado.

%Defina a área de concentração. Se for TCC, deixe comentado.
\areaConcentracao{Desenvolvimento e Sociedade}
%\areaConcentracao{Desenvolvimento e Tecnologias}

% Se houve cotutela, defina:
%\cotutela{Universidade XXX}

% Defina a data da defesa no formato {Dia}{Mês}{Ano}
% Use apenas números! O template transformará em palavras,
% se necessário.
\datadadefesa{25}{7}{2025} % ----------IMPORTANTE------------ %

% Para a versão final defina:
% Repita o nome do Orientador(a) no primeiro avaliador
\avaliadorA{Prof. Dr. Nome do Orientador}{PPG-DTECS/UNIFEI}
\avaliadorB{Profa. Dra. Segunda Avaliadora}{Instituição da segunda avaliadora}
\avaliadorC{Dr. Terceiro Avaliador}{Instituição do terceiro avaliador}
% \avaliadorD{Prof. Dr. Quarto Avaliador}{Instituição do quarto avaliador}
% \avaliadorE{Prof. Dr. Quinto Avaliador}{Instituição do quinto avaliador}
% \avaliadorF{Prof. Dr. Quinto Avaliador}{Instituição do sexto avaliador}

% Descomente na versão final, caso prefira anexar a ficha de um PDF:
%\fichaCatalografica{SeuArquivo.pdf}

% Para deixar uma página em branco no lugar da ficha 
% catalográfica, descomente uma das três linhas a seguir:
%\fichaCatalografica{branco.pdf} % Português
%\fichaCatalografica{white.pdf}  % Inglês
%\fichaCatalografica{blanco.pdf} % Espanhol

% Descomente a linha abaixo se seu trabalho tiver ilustrações
\newif\ifhasillustrations\hasillustrationstrue 

%%%%%%%%%%%%%%%%%%%%%%%%%%%%%%%%%%%%%%%%%%%%%%%%%%%%%%%%%%%%%%%%%%
% --- Inicio das páginas pré-textuais  ---
%%%%%%%%%%%%%%%%%%%%%%%%%%%%%%%%%%%%%%%%%%%%%%%%%%%%%%%%%%%%%%%%%%

% (Não edite aqui. Em caso de erros, entre em contato com o PPG)
\paginasiniciais
  \pagestyle{empty} % páginas pré-textuais sem cabeçalho/rodapé
  \setlength{\oldparindent}{\parindent}
  %\setlength\parindent{0pt}
  \makeatletter
  \let\ps@mystyle\ps@plain
  \let\ps@plain\ps@empty
  \hypersetup{
    pdfcreator   = {Criado com o modelo para teses e dissertações do PPG-DTECS, elaborado por Matheus Mendonça dos Reis, Ms.},
    pdftitle     = {\@titulo},
    pdfauthor    = {\if\@autora\relax\@autor\else\@autora\fi},
    pdfsubject   = {\@monopt\xspace apresentada ao Programa de Pós-Graduação em Desenvolvimento, Tecnologias e Sociedade da Universidade Federal de Itajubá como parte dos requisitos para a obtenção do título de \@degnamept\xspace.}
}
\label{pretextual-end}
%%%%%%%%%%%%%%%%%%%%%%%%%%%%%%%%%%%%%%%%%%%%%%%%%%%%%%%%%%%%%%%%%%
% --- Fim das páginas pré-textuais / Início das textuais ---
%%%%%%%%%%%%%%%%%%%%%%%%%%%%%%%%%%%%%%%%%%%%%%%%%%%%%%%%%%%%%%%%%%

% (Não edite aqui. Em caso de erros, entre em contato com o PPG)
\fimdaspaginasiniciais{
  \cleardoublepage
  \makeatletter
  \let\ps@plain\ps@mystyle
  \makeatother
  % Definição das primeiras páginas dos capítulos
  \fancypagestyle{plain}{
    \fancyhf{}
    \fancyhead[R]{\thepage}
    \renewcommand{\headrulewidth}{0pt}
    \renewcommand{\headheight}{14.49999pt}
  }
  % Definição das demais páginas do texto
  \fancypagestyle{headings}{%
    \fancyhf{}
    \ifprintversion % SE o modo TwoSide estiver ATIVO
      \fancyhead[LE]{\thepage \hspace{1em} \textsc{\nouppercase{\leftmark}}}
      \fancyhead[RO]{\textsc{\nouppercase{\rightmark}} \hspace{1em} \thepage}
    \else % SE for o modo padrão (oneside)
      \fancyhead[L]{\textsc{\nouppercase{\leftmark}}}
      \fancyhead[R]{\thepage}
    \fi
    \renewcommand{\headrulewidth}{1pt}
    \renewcommand{\headheight}{14.49999pt}
  }
  \pagestyle{headings} % ATIVA o estilo de página com numeração
  \setstretch{1.5}
}
%%%%%%%%%%%%%%%%%%%%%%%%%%%%%%%%%%%%%%%%%%%%%%%%%%%%%%%%%%%%%%%%%%

% O corpo da dissertação ou tese começa aqui: (Adapte se necessário)
\include{Capitulos/01-introducao}
\chapter{Referencial Teórico}\label{chp:Referencial}
% O comando a seguir gera um "dummy text". 
% Elimine-o quando escrever sua dissertação.
\lipsum[6]

\chapter{Metodologia}\label{chp:Metodologia}
% O comando a seguir gera um "dummy text". 
% Elimine-o quando escrever sua dissertação.
\lipsum[7]

\chapter{Resultados}\label{chp:Resultados}
% O comando a seguir gera um "dummy text". 
% Elimine-o quando escrever sua dissertação.
\lipsum[8]
\chapter{Discussão}\label{chp:Discussão}
% O comando a seguir gera um "dummy text". 
% Elimine-o quando escrever sua dissertação.
\lipsum[8]
\chapter{Considerações}\label{chp:Considerações}
% O comando a seguir gera um "dummy text". 
% Elimine-o quando escrever sua dissertação.
\lipsum[8]

% O comando condicional \ifturnitin a seguir é importante para 
% preparar o texto para encaminhamento ao Turnitin. 
% NÃO REMOVA!
% O \fi correspondente está antes do \end{document}
\ifturnitin
    \relax
\else 

%%%%%%%%%%%%%%%%%%%%%%%%%%%%%%%%%%%%%%%%%%%%%%%%%%%%%%%%%%%%%%%%%%
% Elementos pós-textuais
%%%%%%%%%%%%%%%%%%%%%%%%%%%%%%%%%%%%%%%%%%%%%%%%%%%%%%%%%%%%%%%%%%

% Define espaçamento simples em cada referência
\begin{singlespacing}

% Adiciona uma linha em branco entre duas referências
\setlength\bibitemsep{10pt}   
%
% Adiciona as referências bibliográficas.
% Mude o título (title), caso o texto seja em inglês 
% ou espanhol.
\printbibliography[heading=bibintoc, % Adiciona no sumário
                   title={Referências bibliográficas} % Nome do Capítulo
                  ]
\end{singlespacing}

% Os anexos, se houver, vêm depois das referências:
\appendix

% O comando a seguir inclui o arquivo apendices.tex
% que contém os apêndices. Observe o comando \appendix
% na linha anterior
\include{Elementos/apendices}
%
%% O comando \fi a seguir é obrigatório para o controle 
%% da opção "turnitin". Não o remova!
% ... (final do seu documento, apêndices, etc.) ...
\fi % Este é o \fi do bloco do Turnitin

% Exibe a "quarta capa" conhecida como capa de trás
\quartacapa

%%%%%%%%%%%%%%%%%%%%%%%%%%%%%%%%%%%%%%%%%%%%%%%%%%%%%%%%%%%%%%%%%%
\end{document} % Prontinho! Quebre a perna!
%%%%%%%%%%%%%%%%%%%%%%%%%%%%%%%%%%%%%%%%%%%%%%%%%%%%%%%%%%%%%%%%%%